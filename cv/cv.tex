\documentclass[margin]{res}  
\usepackage{hyperref}
\usepackage{enumitem}
\usepackage{paralist}
\usepackage{comment}
\usepackage[dvipsnames]{xcolor}

\newcommand\myshade{85}
\colorlet{mylinkcolor}{blue}
\definecolor{chestnut}{rgb}{0.8, 0.36, 0.36}
\hypersetup{
  linkcolor  = mylinkcolor!\myshade!black,
  citecolor  = mylinkcolor!\myshade!black,
  urlcolor   = mylinkcolor!\myshade!black,
  colorlinks = true,
}

\newcommand{\field}[2]{\noindent \textbf{#1} \hfill #2 \\}
\newcommand{\uf}[2]{\noindent \textbf{#1} \hfill \url{#2} \\}
\newcommand{\uff}[2]{\item[--] #1 \\ \url{#2}}
\newcommand{\alphabeticalorder}[0]{\ensuremath {^{\alpha \beta}}}
\newcommand{\specialurl}[2]{\href {#2} {\texttt{[#1]}}}
\newcommand{\preprint}[1]{\specialurl {preprint} {#1}}
\newcommand{\code}[1]{\specialurl {code} {#1}}
\newcommand{\data}[1]{\specialurl {data} {#1}}
\newcommand{\talk}[1]{\specialurl {talk} {#1}}
\newcommand{\paper}[1]{\specialurl {paper} {#1}}
\newcommand{\authorref}[1]{\underline {\textbf{#1}}}
\newcommand{\authorme}{\authorref{Papachristou, Marios}}

% Increase text height
\textheight=680pt

\begin{document}

%------------------------------------X-------------------------------------------
%    NAME AND ADDRESS SECTION
%-------------------------------------------------------------------------------
\name{\textsc{Marios A. Papachristou} (Last Updated: \today)}

% Note that addresses can be used for other contact information:
% -phone numbers
% -email addresses
% -linked-in profile

% \address{\textsc{GitHub}: \href{http://github.com/papachristoumarios}{\nolinkurl{papachristoumarios}}
% \\ 
%\textsc{Address: Archadjikaki 5, Athens, Greece, 17237 }
%}
% \address{\textsc{email}: \url{papachristoumarios@cs.cornell.edu} % \\ \textsc{Mobile:} (+30) 6979614463
%}

% Uncomment to add a third address
%\address{Address 3 line 1\\Address 3 line 2\\Address 3 line 3}
%-------------------------------------------------------------------------------


\begin{resume}

\section{Contact \\ Information} 
E-mail \hfill \url{papachristoumarios@cs.cornell.edu} \\
GitHub \hfill \href{http://github.com/papachristoumarios}{\nolinkurl{papachristoumarios}} \\
Office \hfill 302 Gates Hall, Cornell University \\
Google Scholar \hfill \specialurl{profile}{https://scholar.google.gr/citations?user=T12JO3MAAAAJ&hl=en}


\section{Interests} Data Science (information networks, statistical methods, fairness in networks)

\section{Education}
\field {Cornell University} {\emph{2020 -- exp. 2026}}
Ph.D. in Computer Science (\emph{GPA: 4.0}), Minor: \emph{Applied Math}
\begin{compactitem}
\item[--] \emph{Advisor:} Jon Kleinberg, \emph{Committee:} Jon Kleinberg, Emma Pierson, Sid Banerjee 
\item[--] \emph{Coursework:} Analysis of Algorithms, Optimal Stopping, Information Networks, Numerical Methods for Data Science, Design of Online Marketplaces
\end{compactitem} 
 
\field{National Technical University of Athens}  {\emph{2015--2020}} 
Diploma (5-year joint degree, 300 ECTS) Electrical \& Computer Engineering (ECE)
\begin{compactitem}
\item[--] \emph{GPA:} 3.8/4.0 (top 1\%)
\item[--] \emph{Major:} Computer Science (\emph{Major GPA:} 3.8/4.0)
\item[--] \emph{Thesis: ``Stochastic Opinion Dynamics for Interest Prediction in Online Social Networks''. Advisor: Dimitris Fotakis. Grade: 4.0/4.0}
\end{compactitem} 
%\field{1st General Lyceum of Hymettus} {\emph{2012--2015}} 
%    Apolytirion, Valedictorian, GPA: 19.8/20.0, top 2\% in National Exams (18,971/20,000)


\section{Publications}

\begin{compactenum}
    %\item[1.] \underline{\textbf{Papachristou, Marios}} and Dimitris Fotakis. "Learning user interests from Influencers in Online Social Networks". \emph{Work in progress}.
    
   \item  \authorme, and Kleinberg, Jon. ``Allocating Stimulus Checks in Times of Crisis''. 2021, \emph{WWW '22 (To appear) \& EAAMO '21 (Poster).} \preprint{https://arxiv.org/abs/2106.07560} \code{https://github.com/papachristoumarios/financial-contagion}     
   \item \authorme ``Sublinear Domination and Core--periphery Networks''. 2021,  \emph{Scientific Reports (Nature).}  \preprint{https://arxiv.org/abs/2103.03135} \code{https://bit.ly/3wKNGI0} \paper{http://www.nature.com/articles/s41598-021-94105-8}
    \item \alphabeticalorder Chalkis, Apostolos, Fisikopoulos Vissarion, \authorme, and Tsigaridas, Elias. ``Truncated Log-concave Sampling with Reflective Hamiltonian Monte Carlo''. 2021 \preprint{https://arxiv.org/abs/2102.13068} \code{https://github.com/GeomScale/volume_approximation}
    \item \authorme, and Fotakis, Dimitris. ``Stochastic Opinion Dynamics for User Interest Prediction in Online Social Networks''. 2020, \emph{Submitted.}  \preprint{https://www.researchgate.net/publication/353006940_Stochastic_Opinion_Dynamics_for_Interest_Prediction_in_Social_Networks}
    \item \authorme. ``Software clusterings with vector semantics and the call graph''. ESEC/FSE 2019.  \paper{https://dl.acm.org/citation.cfm?id=3342483} \code{https://github.com/papachristoumarios/sade} \data{http://doi.org/10.5281/zenodo.2652487}
    \item Kostakis, Vasilis, and \authorme. ``Commons-based peer production and digital fabrication: The case of a RepRap-based, Lego-built 3D printing-milling machine''. Telematics and Informatics,  \paper{https://bit.ly/2JRoisV} 
    
\end{compactenum}

\alphabeticalorder  = alphabetical order, * = equal contribution

%\section{Working Papers}
%
%\begin{compactitem}
%
%\item[1.] ``Stochastic Opinion Dynamics for Interest Prediction in Online Social Networks''. \emph{with D. Fotakis. Submitted for publication.}
%\item[2.] ``Opinion Dynamics with Fairness Constraints''. \emph{with J. Kleinberg. Ongoing.}
%\item[3.] ``Sampling from truncated high-dimensional log-concave densities with Hamiltonian Monte Carlo: Mixing Time Analysis, Software, and Applications''. \emph{with A. Chalkis, V. Fisikopoulos, E. Tsigaridas. Ongoing.}
%
%\end{compactitem}

\section{Research \\ Experience}

\textbf{Cornell University (Graduate Research Assistant)} \hfill \emph{May 2021 --}
\begin{compactitem}
	\item[--] Thesis-related research
\end{compactitem}

\textbf{Business Analytics Lab (Undergraduate Researcher)} \hfill \emph{2018 -- 2020}
\begin{compactitem}
\item[--] Conduct research on Machine Learning on Software Engineering: Source Code Embeddings, Software Clusterings, Layering Violations
%\begin{compactitem}
%\item Source Code Embeddings
%\item Software Clustering
%\item Layering Violations
%\end{compactitem}
\item[--] Research funded by the \textsc{Crossminer} project, supported by \emph{Horizon 2020} grant
\item[--] Research Advisor: Prof. Diomidis Spinellis
\end{compactitem}

\textbf{Hellenic Center for Marine Research} \hfill \emph{June 2014 -- August 2014}
% \begin{compactitem}
% \item[--] Develop a system for marine species identification and morphometric analysis using features and ORB, SIFT and SURF descriptors in OpenCV \href{https://github.com/papachristoumarios/triton-fpr}{\nolinkurl{[repository]}}
% \end{compactitem}



\textbf{P2P Lab (Remote Research Associate)} \hfill \emph{2013--2014}

%\begin{compactitem}
% \item[--] Publish self-made Lego 3D Printing-Milling machine at age 16 with Prof. Vasilis Kostakis \href{https://github.com/papachristoumarios/lego-mindstorms-3d-printing-milling-machine}{\nolinkurl{[repository]}}
% \end{compactitem}

 
\section{Professional \\ Experience}
\field{Google Summer of Code 2020 (GeomScale)}  {\emph{June 2020--}}
\begin{compactitem}
\item[--] \emph{Project: Sampling from high-dimensional log-concave densities} 
\item[--] Develop software for the efficient sampling from high-dimensional log-concave densities using first-order oracles in (un)-truncated settings \code{https://GeomScale/volume_approximation}
\item[--] Presented at \href{https://www.youtube.com/watch?v=P7YfC8Nn6sY}{PyData Global 2020} and the \href{https://www.cs.utah.edu/~jeffp/WaGoML/index.html}{Workshop on Geometry and ML} at SoCG 2021. 
\item[--] The software includes ODE and SDE solver and samplers for convex body domains (convex polytopes).
\end{compactitem}

\field{Google Summer of Code 2018 (GFOSS-OTA)}  {\emph{April 2018--September 2018}}
\begin{compactitem}
\item[--] Developed a fully functional project for text mining, cross-linking and automated codification of Greek Legislation using Natural Languaging Processing \& Data Mining Methods and Practices.  \code{https://github.com/eellak/gsoc2018-3gm} \data{https://archive.org/details/greekgovernmentgazette} \talk{https://www.youtube.com/watch?v=_UIGsy85Ehw}

% \href{https://3gm.ellak.gr}{\nolinkurl{3gm.ellak.gr}} \href{https://github.com/eellak/gsoc2018-3gm}{\nolinkurl{[repository]}}
\begin{compactitem}
\item[--] Cross-linking into Dynamic Graphs, Automated Codification
\item[--] Topic Modeling, Embeddings, Ranking (PageRank)
\item[--] Greek Legislative Texts Internet Archive Collection 
\end{compactitem}
\end{compactitem}

\field{Ratle (Co-founder)} {\emph{October 2017--October 2018}}
\begin{compactitem}
\item[--] Cashierless shopping system start-up. 
% \item[--] Bluetooth Signal Denoising with Kalman Filters, Product tracking inside stores, Enterprise application for Business Analytics
% \item[--] Tested at real stores % \href{https://www.youtube.com/watch?v=XemEQaNxL3Q&feature=youtu.be}{\nolinkurl{[video]}}
% \item[--] Discontinued due to heavy academic load and work vs. academia quandary
\end{compactitem}


\section{Teaching \\ Assistantship \\ Experience} 
\textbf{Undergraduate Teaching Assistant}
\begin{compactitem}
    \item[--] Discrete Mathematics (4th Semester) \hfill \emph{Spring 2017}
    \item[--] Programming Techniques (2nd Semester) \hfill \emph{Spring 2016}
    \item[--] Introduction to Computer Programming (1st Semester) \hfill \emph{Fall 2016, Fall 2017}
\end{compactitem}



% \section{Languages} Greek (Native), English (C2), German (B1)


\begin{comment}
\section{Data \\ Analysis \& \\ Visualization \\ Tools} 
\begin{compactitem}
\item[--] Python 
\item[--] Bash 
\item[--] MATLAB 
\item[--] GNU Octave 
\item[--] R
\item[--] Matplotlib 
\item[--] GNUPlot 
\item[--] Pandas
\item[--] Seaborn
\item[--] MySQL 
\item[--] MongoDB 
\end{compactitem}
\end{comment}

\section{Technical \\ Skills}

\field{Programming Languages}  {Python, C, C++,, R, SQL  } 
\field{Machine Learning Frameworks}  {PyTorch,  Sklearn} 
\field{Data Analysis and Visualization} {matplotlib, MATLAB, pandas, Bash } 
\field{Scientific Programming} {NumPy, SciPy, NetworkX, MATLAB} 
\field{Databases} {MySQL, MongoDB}

\section{Honors \& \\ Awards} 
\begin{compactitem}
	\item[--] Cornell New Student Fellowship (first-year). Cornell University. 
	\item[--] Thomaidion Award. Publication during undergraduate studies (2019). NTUA. 
    \item[--] Programming Competition \emph{IEEEXtreme 12.0}. Ranked 48th out of 4,000 Worldwide (top 1\%), 1st in Greece, 13th in Europe (PComplete Team)
    \item[--] 4th (out of 93) Place in \emph{International Space Engineering  Competition at Texas} (CanSat) organized by NASA Goddard and the Americal Astronautical Society
    \item[--] 1st Award at \emph{``Crowdhackathon Fintech \#2''} for developing a cashierless system for retail shops to eliminate queues by National Bank of Greece
    \item[--] 2nd Award at the \emph{``be finnovative 2.0 accelerator''} for developing a cashierless system for retail shops to eliminate queues by National Bank of Greece
    \item[--] ESEC/FSE 2019 ACM Student Research Competition Finalist
    \item[--] 2nd Place at the \emph{``ECESCON9''} Hackathon 
    \item [--] \emph{``Touramanoglu''} Scholarship by Municipality of Helioupolis \& Hymettus
    \item [--] \emph{``The Great Moment of Education''} Scholarship by Eurobank EFG 

\end{compactitem}

% \section{Memberships} IEEE Member, ACM Member

\section{Volunteering} 
\begin{compactitem}
    \item[--] Teaching algorithms courses in National Competition for Informatics Camp 
    \item[--] Google Summer of Code Mentor (2019, w/ GFOSS-OTA, 2021 w/GeomScale)
    \item[--] Member of TEDx NTUA 2018 Organizing Committee, IT Team 
    \item[--] Contributor to open-source projects to the \href{https://geomscale.github.io/}{GeomScale} organization.
    \item [--] \emph{Reviewer.} \href{https://joss.theoj.org/}{JOSS}, \href{https://sites.google.com/view/whmd2021}{WHMD@NeurIPS `21}, \href{https://facctconference.org/2022}{FAccT `22}

\end{compactitem}

\section{References}

\href{mailto:kleinberg@cornell.edu}{Jon Kleinberg} (PhD Advisor), 
\href{mailto:fotakis@cs.ntua.gr}{Dimitris Fotakis}  (U/G Advisor),
\href{mailto:dds@aueb.gr}{Diomidis Spinellis} (Research Advisor), 
\href{mailto:kostakis@law.harvard.edu}{Vasilis Kostakis}  (Harvard BKC) 



\end{resume}

\end{document}
