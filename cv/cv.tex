\documentclass[margin, 10pt]{res}  
\usepackage[hidelinks]{hyperref}
\usepackage{enumitem}
\usepackage{paralist}
\usepackage{comment}
\usepackage[dvipsnames]{xcolor}

\newcommand\myshade{85}
\colorlet{mylinkcolor}{blue}
\definecolor{chestnut}{rgb}{0.8, 0.36, 0.36}
\newcommand{\marios}[1]{\textcolor{chestnut}{Marios: #1}}
\hypersetup{
  linkcolor  = mylinkcolor!\myshade!black,
  citecolor  = mylinkcolor!\myshade!black,
  urlcolor   = mylinkcolor!\myshade!black,
  colorlinks = true,
}



\usepackage{helvet}
\newcommand{\field}[2]{\noindent \textbf{#1} \hfill #2 \\}
\newcommand{\uf}[2]{\noindent \textbf{#1} \hfill \url{#2} \\}
\newcommand{\uff}[2]{\item[--] #1 \\ \url{#2}}
\newcommand{\alphabeticalorder}[0]{\ensuremath  {^\flat}}
\newcommand{\specialurl}[2]{\href { #2 } {\nolinkurl{[#1]}}}
\newcommand{\preprint}[1]{\specialurl {preprint} {#1}}
\newcommand{\code}[1]{\specialurl {code} {#1}}
\newcommand{\data}[1]{\specialurl {data} {#1}}
\newcommand{\paper}[1]{\specialurl {paper} {#1}}
\newcommand{\authorref}[1]{\underline {\textbf{#1}}}
\newcommand{\authorme}{\authorref{Papachristou, Marios}}

% Increase text height
\textheight=680pt

\begin{document}

%------------------------------------X-------------------------------------------
%    NAME AND ADDRESS SECTION
%-------------------------------------------------------------------------------
\name{\textsc{Marios A. Papachristou}}

% Note that addresses can be used for other contact information:
% -phone numbers
% -email addresses
% -linked-in profile

% \address{\textsc{GitHub}: \href{http://github.com/papachristoumarios}{\nolinkurl{papachristoumarios}}
% \\ 
%\textsc{Address: Archadjikaki 5, Athens, Greece, 17237 }
%}
% \address{\textsc{email}: \url{papachristoumarios@cs.cornell.edu} % \\ \textsc{Mobile:} (+30) 6979614463
%}

% Uncomment to add a third address
%\address{Address 3 line 1\\Address 3 line 2\\Address 3 line 3}
%-------------------------------------------------------------------------------


\begin{resume}

\section{Contact \\ Information} 
E-mail \hfill \url{papachristoumarios@cs.cornell.edu} \\
GitHub \hfill \href{http://github.com/papachristoumarios}{\nolinkurl{papachristoumarios}} \\
Office \hfill 302 Gates Hall, Cornell University \\
Google Scholar \hfill \specialurl{profile}{https://scholar.google.gr/citations?user=T12JO3MAAAAJ&hl=en}


\section{Interests} Data Science (information networks, statistical methods, fairness in networks)

\section{Education}
\field {Cornell University} {\emph{2020 -- exp. 2026}}
Ph.D. in Computer Science. Concentration on Data Science
\begin{compactitem}
\item[--] \emph{Advisor:} Prof. Jon Kleinberg
\item[--] \emph{GPA:} 4.0
\item[--] \emph{Coursework:} Analysis of Algorithms, Optimal Stopping, Information Networks, Numerical Methods for Data Science
\end{compactitem} 
 
\field{National Technical University of Athens}  {\emph{2015--2020}} 
Diploma (5-year joint degree, 300 ECTS) Electrical \& Computer Engineering (ECE)
\begin{compactitem}
\item[--] \emph{GPA:} 3.8/4.0 (top 1\%)
\item[--] \emph{Major:} Computer Science (\emph{Major GPA:} 3.8/4.0)
\item[--] \emph{Thesis: ``Stochastic Opinion Dynamics for Interest Prediction in Online Social Networks''. Advisor: Dimitris Fotakis. Grade: 4.0/4.0}
\end{compactitem} 
%\field{1st General Lyceum of Hymettus} {\emph{2012--2015}} 
%    Apolytirion, Valedictorian, GPA: 19.8/20.0, top 2\% in National Exams (18,971/20,000)


\section{Publications}

\begin{compactenum}
    %\item[1.] \underline{\textbf{Papachristou, Marios}} and Dimitris Fotakis. "Learning user interests from Influencers in Online Social Networks". \emph{Work in progress}.
    
   \item  \authorme, and Kleinberg, Jon. ``Allocating Stimulus Checks in Times of Crisis''. 2021, \emph{Submitted.} \preprint{https://arxiv.org/abs/2106.07560} \code{https://github.com/papachristoumarios/financial-contagion}     
   \item \authorme ``Sublinear Domination and Core--periphery networks''. 2021,  \emph{Scientific Reports (Nature).}  \preprint{https://arxiv.org/abs/2103.03135} \code{https://bit.ly/3wKNGI0}
    \item \alphabeticalorder Chalkis, Apostolos, Fisikopoulos Vissarion, \authorme, and Tsigaridas, Elias. ``Truncated Log-concave Sampling with Reflective Hamiltonian Monte Carlo''. 2021 \preprint{https://arxiv.org/abs/2102.13068} \code{https://github.com/GeomScale/volume_approximation}
    \item \authorme, and Fotakis, Dimitris. ``Stochastic Opinion Dynamics for User Interest Prediction in Online Social Networks''. 2020, \emph{Submitted.}  \preprint{https://www.researchgate.net/publication/353006940_Stochastic_Opinion_Dynamics_for_Interest_Prediction_in_Social_Networks}
    \item \authorme. ``Software clusterings with vector semantics and the call graph''. ESEC/FSE 2019.. \paper{https://dl.acm.org/citation.cfm?id=3342483} \code{https://github.com/papachristoumarios/sade} \data{http://doi.org/10.5281/zenodo.2652487}
    \item Kostakis, Vasilis, and \authorme. ``Commons-based peer production and digital fabrication: The case of a RepRap-based, Lego-built 3D printing-milling machine''. Telematics and Informatics,  \paper{https://bit.ly/2JRoisV} 
    
\end{compactenum}

\alphabeticalorder  = alphabetical order.

%\section{Working Papers}
%
%\begin{compactitem}
%
%\item[1.] ``Stochastic Opinion Dynamics for Interest Prediction in Online Social Networks''. \emph{with D. Fotakis. Submitted for publication.}
%\item[2.] ``Opinion Dynamics with Fairness Constraints''. \emph{with J. Kleinberg. Ongoing.}
%\item[3.] ``Sampling from truncated high-dimensional log-concave densities with Hamiltonian Monte Carlo: Mixing Time Analysis, Software, and Applications''. \emph{with A. Chalkis, V. Fisikopoulos, E. Tsigaridas. Ongoing.}
%
%\end{compactitem}

\section{Research \\ Experience}

\textbf{Cornell Universtiy (Graduate Research Assistant)} \hfill \emph{May 2021 --}
\begin{compactitem}
	\item[--] Thesis-related research
\end{compactitem}

\textbf{Business Analytics Lab (Undergraduate Researcher)} \hfill \emph{2018 -- 2020}
\begin{compactitem}
\item[--] Conduct research on Machine Learning on Software Engineering: Source Code Embeddings, Software Clusterings, Layering Violations
%\begin{compactitem}
%\item Source Code Embeddings
%\item Software Clustering
%\item Layering Violations
%\end{compactitem}
\item[--] Research funded by the \textsc{Crossminer} project, supported by \emph{Horizon 2020} grant
\item[--] Research Advisor: Prof. Diomidis Spinellis
\end{compactitem}

\textbf{Hellenic Center for Marine Research} \hfill \emph{June 2014 -- August 2014}
% \begin{compactitem}
% \item[--] Develop a system for marine species identification and morphometric analysis using features and ORB, SIFT and SURF descriptors in OpenCV \href{https://github.com/papachristoumarios/triton-fpr}{\nolinkurl{[repository]}}
% \end{compactitem}



\textbf{P2P Lab (Remote Research Associate)} \hfill \emph{2013--2014}

%\begin{compactitem}
% \item[--] Publish self-made Lego 3D Printing-Milling machine at age 16 with Prof. Vasilis Kostakis \href{https://github.com/papachristoumarios/lego-mindstorms-3d-printing-milling-machine}{\nolinkurl{[repository]}}
% \end{compactitem}

 
\section{Professional \\ Experience}
\field{Google Summer of Code 2020 (GeomScale)}  {\emph{June 2020--}}
\begin{compactitem}
\item[--] \emph{Project: Sampling from high-dimensional log-concave densities} 
\item[--] Develop software for the efficient sampling from high-dimensional log-concave densities using first-order oracles in (un)-truncated settings \code{https://GeomScale/volume_approximation}
%{\nolinkurl{[github]}} \href{https://papachristoumarios.github.io/2020/07/21/Sampling-from-high-dimensional-truncated-log-concave-densities-with-volesti/}{\nolinkurl{[post]}} \href{https://www.youtube.com/watch?v=P7YfC8Nn6sY}{\nolinkurl{[talk]}}
\item[--] The software includes ODE and SDE solver and samplers for convex body domains (convex polytopes).
\end{compactitem}

\field{Google Summer of Code 2018 (GFOSS-OTA)}  {\emph{April 2018--September 2018}}
\begin{compactitem}
\item[--] Developed a fully functional project for text mining, cross-linking and automated codification of Greek Legislation using Natural Languaging Processing \& Data Mining Methods and Practices.  \code{https://github.com/eellak/gsoc2018-3gm}. Internet Archive Dataset: \data{https://archive.org/details/greekgovernmentgazette} 

% \href{https://3gm.ellak.gr}{\nolinkurl{3gm.ellak.gr}} \href{https://github.com/eellak/gsoc2018-3gm}{\nolinkurl{[repository]}}
\begin{compactitem}
\item[--] Cross-linking into Dynamic Graphs, Automated Codification
\item[--] Topic Modeling, Embeddings, Ranking (PageRank)
\item[--] Greek Legislative Texts Internet Archive Collection 
\end{compactitem}
\end{compactitem}

\field{Ratle (Co-founder)} {\emph{October 2017--October 2018}}
\begin{compactitem}
\item[--] Cashierless shopping system.
\item[--] Bluetooth Signal Denoising with Kalman Filters, Product tracking inside stores, Enterprise application for Business Analytics
\item[--] Tested at real stores % \href{https://www.youtube.com/watch?v=XemEQaNxL3Q&feature=youtu.be}{\nolinkurl{[video]}}
% \item[--] Discontinued due to heavy academic load and work vs. academia quandary
\end{compactitem}


\section{Teaching \\ Assistantship \\ Experience} 
\textbf{Undergraduate Teaching Assistant}
\begin{compactitem}
    \item[--] Discrete Mathematics (4th Semester) \hfill \emph{Spring 2017}
    \item[--] Programming Techniques (2nd Semester) \hfill \emph{Spring 2016}
    \item[--] Introduction to Computer Programming (1st Semester) \hfill \emph{Fall 2016, Fall 2017}
\end{compactitem}



% \section{Languages} Greek (Native), English (C2), German (B1)


\begin{comment}
\section{Data \\ Analysis \& \\ Visualization \\ Tools} 
\begin{compactitem}
\item[--] Python 
\item[--] Bash 
\item[--] MATLAB 
\item[--] GNU Octave 
\item[--] R
\item[--] Matplotlib 
\item[--] GNUPlot 
\item[--] Pandas
\item[--] Seaborn
\item[--] MySQL 
\item[--] MongoDB 
\end{compactitem}
\end{comment}

\section{Technical \\ Skills}

\field{Programming Languages}  {Python, C/C++, Java, JavaScript/TypeScript, Bash, OCaml, PHP, R, SQL  } 
\field{Machine Learning Frameworks}  {PyTorch, Pyro, Tensorflow Keras, Sklearn, spaCy } 
\field{Data Analysis and Visualization} {matplotlib, MATLAB, pandas, Bash } 
\field{Scientific Programming} {SAGE, Octave, MATLAB, Scilab, NumPy, SciPy, OpenCV, NetworkX} 
\field{Databases} {MySQL, MongoDB, PostgreSQL, Firebase}
%\field{Web Frameworks} {Django, AngularJS, Ionic, Laravel PHP}
\field{Version Control} {Git, Subversion}
%\field{Typesetting} {\LaTeX, WYSIWYG Typewriting}
%field{Operating Systems} {GNU/Linux}
\section{Honors \& \\ Awards} 
\begin{compactitem}
	\item[--] Cornell New Student Fellowship (first-year). Cornell University. 
    \item[--] Programming Competition \emph{IEEEXtreme 12.0}. Ranked 48th out of 4,000 Worldwide (top 1\%), 1st in Greece, 13th in Europe (PComplete Team)
    \item[--] 4th (out of 93) Place in \emph{International Space Engineering  Competition at Texas} (CanSat) organized by NASA Goddard and the Americal Astronautical Society
    \item[--] 1st Award at \emph{``Crowdhackathon Fintech \#2''} for developing a cashierless system for retail shops to eliminate queues by National Bank of Greece
    \item[--] 2nd Award at the \emph{``be finnovative 2.0 accelerator''} for developing a cashierless system for retail shops to eliminate queues by National Bank of Greece
    \item[--] ESEC/FSE 2019 ACM Student Research Competition Finalist
    \item[--] 2nd Place at the \emph{``ECESCON9''} Hackathon 
    \item [--] \emph{``Touramanoglu''} Scholarship by Municipality of Helioupolis \& Hymettus
    \item [--] \emph{``The Great Moment of Education''} Scholarship by Eurobank EFG 

\end{compactitem}

% \section{Memberships} IEEE Member, ACM Member

\section{Volunteering} 
\begin{compactitem}
    \item[--] Teaching algorithms courses in National Competition for Informatics Camp 
    \item[--] Google Summer of Code Mentor (2019, w/ GFOSS-OTA, 2021 w/GeomScale)
    \item[--] Member of TEDx NTUA 2018 Organizing Committee, IT Team 
    \item[--] Contributor to open-source projects. Member of the GFOSS-OTA, and GeomScale open-source organizations.

\end{compactitem}

\section{References \\ available \\ upon request}

Jon Kleinberg (PhD Advisor, \texttt{kleinberg@cornell.edu}), 
Dimitris Fotakis  (U/G Advisor, \texttt{fotakis@cs.ntua.gr}),
Diomidis Spinellis  (Research Advisor, \texttt{dds@aueb.gr}), 
Vasilis Kostakis  (Harvard BKC, \texttt{kostakis@law.harvard.edu}) 

%\section{Hobbies}
%Cycling, Bike Restoration, Competitive Programming, Basketball, Reading, Drama

\section{Last Updated} \today

\begin{comment}

\section{Selected \\ Projects \& Published \\ Software} 

\field{SADE} {\emph{August 2018 --}}
This project aims to perform software clusterings using vector semantics and the call graphs. The project had successful results on recovering clusters of the Linux Kernel.\\
Repository: \url{https://github.com/papachristoumarios/sade}

\field{3gm -- Google Summer of Code} {\emph{April 2018 -- }}
This project aims to provide an automated codex of the most recent versions of each law in Greek Legislation via NLP methods and practices. Results are published at \url{3gm.ellak.gr}. \\ Repository: \url{https://github.com/eellak/gsoc2018-3gm}

\field{Ratle} {\emph{November 2017 -- October 2018}}
I was co-founder and software engineer of a fin-tech startup called Ratle. We had implemented a fully-working and automated contactless transaction system (\url{https://goo.gl/UymCdN}) that aims to eliminate the need for the cashiers and the struggle of time-consuming waiting queues. The activity of the startup is currently paused due to a heavy school load. There are
no plans for restarting in the foreseeable future.

\field{SignGlove} {\emph{March 2016 (Hackathon Contest)}}
SignGlove is a gesture glove targeted for people with disabilities. It features a sign language translator glove capable of transforming sign language to speakable words which were built during the ECESCON 9 Hackathon. \\ Repository: \url{https://github.com/papachristoumarios/SignGlove}

\field{TritonFPR} {\emph{July 2014-August 2014 (Research Project at Hellenic Centre for Marine Research)}}
Triton FPR is a Fish Pattern Recognition Project which aims to identify captured species and provide the users and researchers with the proper information. The user captures a photo of the desired specimen and requests identification. Then, the computer attempts identification by looking at a pre-generated database being acquired during previous research activities on already acquainted species. Finally, it performs morphometric analysis on the selected specimen. It was developed using Python, NumPy and OpenCV \\ Repository: \url{https://github.com/papachristoumarios/triton-fpr}

\field{LEGO 3D Printing-Milling Machine} {\emph{Approx. 2010 (Freelance Project)}}
This is a project in which a special (reverse-engineered) RepRap 3D printer and milling machine. Using LEGO as modular components makes the machine is made hybrid with the capability of placing a specialized extruder instead of the milling bit. This project also led to a paper publication at Elsevier’s Telematics \& Informatics Journal at age 16 (see [1]). \\ Repository:  \url{https://goo.gl/rdzkkM}

\field{CScout} {\emph{Aug. 2018-- (Contributor)}} Contribute to CSCout functionality.
Contributions can be found here: \url{https://github.com/dspinellis/cscout/pulls?q=is%3Apr+author%3Apapachristoumarios+is%3Aclosed}


\section{Links} 

\begin{compactitem}
    \uff{School Curriculum} {https://www.ece.ntua.gr/media/595/odigos_proptyxiakwn_spoydwn.pdf}
    \uff{GitHub Profile} {https://github.com/papachristoumarios}
    \uff{Google Summer of Code 2018 Project Repository (3gm)} {https://summerofcode.withgoogle.com/archive/2018/projects/5104033505214464}
    \uff{Google Summer of Code 2018 Application (3gm)} {https://3gm.ellak.gr}
    
    \uff{Prize in International Space Engineering Competition} {https://bit.ly/2XZ1jWH}
    \uff{Fintech Hackathon} {https://bit.ly/2JYxtYM}
    \uff{be finnovative 2.0 Accelerator} {https://www.youtube.com/watch?v=Q-4fIfkIlqo}
    %\uff{}
\end{compactitem}
\end{comment}

\end{resume}
\(\)\end{document}
